%!TEX program = xelatex

\documentclass[compress]{beamer}
%--------------------------------------------------------------------------
% Common packages
%--------------------------------------------------------------------------
\usepackage[english]{babel}
\usepackage{pgfpages} % required for notes on second screen
\usepackage{graphicx}

\usepackage{multicol}

\usepackage{tabularx,ragged2e}
\usepackage{booktabs}

\usepackage{listings}
\lstset{ %
language=[LaTeX]TeX,
basicstyle=\normalsize\ttfamily,
keywordstyle=,
numbers=left,
numberstyle=\tiny\ttfamily,
stepnumber=1,
showspaces=false,
showstringspaces=false,
showtabs=false,
breaklines=true,
frame=tb,
framerule=0.5pt,
tabsize=4,
framexleftmargin=0.5em,
framexrightmargin=0.5em,
xleftmargin=0.5em,
xrightmargin=0.5em
}


%--------------------------------------------------------------------------
% Load theme
%--------------------------------------------------------------------------
\usetheme{hri}

\usepackage{dtklogos} % must be loaded after theme
\usepackage{tikz}
\usetikzlibrary{mindmap,backgrounds,positioning}

\graphicspath{{figs/}}

%--------------------------------------------------------------------------
% General presentation settings
%--------------------------------------------------------------------------
\title{The Robot Operating System}
\subtitle{High-Altitude Overview of ROS}
\date{\today}
\author{Séverin Lemaignan}
\institute{Centre for Robotics and Neural Systems\\ {\Medium Plymouth University}}

%--------------------------------------------------------------------------
% Notes settings
%--------------------------------------------------------------------------
%\setbeameroption{show notes on second screen}

\begin{document}
%--------------------------------------------------------------------------
% Titlepage
%--------------------------------------------------------------------------

\maketitle

%--------------------------------------------------------------------------
% Table of contents
%--------------------------------------------------------------------------
%\section*{Overview}
%\begin{frame}{Overview}
%	% hideallsubsections ist empfehlenswert für längere Präsentationen
%	\tableofcontents[hideallsubsections]
%\end{frame}

%--------------------------------------------------------------------------
% Content
%--------------------------------------------------------------------------
\section{ROS is not an operating system}

\begin{frame}{Instead, ROS is...}
    \begin{itemize}
        \item<1-> A fairly simple message passing system designed with robotics in
            mind
        \item<2-> An API to this system (in several languages -- C++ and Python are
            1st tier)
        \item<3> \Medium{A middleware?}
        \item<4-> A set of conventions to write and package robotic softwares
        \item<5-> A set of standard message types that facilitate interoperability
            between modules
        \item<6-> A set of tools to run and monitor the nodes
        \item<7-> A very large academic community leading to a library of thousands of nodes
    \end{itemize}
\end{frame}

\begin{frame}{ROS Ecosystem}
    \centering
    \resizebox{0.9\textwidth}{!}{%
        \vspace*{4cm}
        \begin{tikzpicture}

            \path[small mindmap,
                  level 1 concept/.append style={sibling angle=360/5}, 
                  level 2 concept/.append style={sibling angle=60}, 
                  concept color=hriWarmGreyLight,text=hriWarmGreyDark]

            node[concept] {\Medium ROS}
            [clockwise from=-180]
            child[concept color=hriSec1Dark,text=white] { node[concept]{Middleware} 
                [clockwise from=-120]
                child[concept color=hriSec1CompDark,text=white] { node[concept]{Standard Interfaces} }
                child[concept color=hriSec3CompDark,text=white] { node[concept]{Nodes Management} }
                child[concept color=hriSec2Dark,text=white] { node[concept]{IPC} }
            }
            child[concept color=hriSec3Comp,text=white] { node[concept] {Standards and Conventions} }
            child[concept color=hriSec2CompDark,text=white] { node[concept]{Software engineering infrastructure} }
            child[concept color=hriSec3Dark,text=white] { node[concept] {Large software library} 
                [clockwise from=0]
                child[concept color=hriSec2Dark,text=white] { node[concept] {Satellite libraries} }
            }
            child[concept color=hriSec3CompDark,text=white] { node[concept] {Tooling} };
        \end{tikzpicture}
    }
\end{frame}

\begin{frame}{}
    This being clarified...
\end{frame}

\section{What is it in there to be excited about?}

\begin{frame}[containsverbatim]{}

\begin{shcode}
$ roscore
$ rosrun rviz rviz
$ roslaunch openni_launch openni.launch
\end{shcode}

\end{frame}


\imageframe{rviz0}

\begin{frame}[containsverbatim]{}

\begin{shcode}
$ rosrun attention_tracker estimate 
                         image:=/camera/rgb/image_color
\end{shcode}

\end{frame}

\imageframe{rviz}

\begin{frame}{}
\end{frame}


\section{Into the details: the key concepts}

\section{Satellite libraries}

\begin{frame}{TF}
\end{frame}

\begin{frame}{OpenCV}
\end{frame}

\begin{frame}{PCL}
\end{frame}


\section{Tooling}

\begin{frame}{Tools}
    \begin{table}[]
        \begin{tabularx}{\linewidth}{l>{\raggedright}X}
            \toprule
            \texttt{rviz} & versatile 2D/3D visualization \tabularnewline
            \texttt{rosconsole} & Centralized logging \tabularnewline
            \texttt{rosbag} & Record and replay messages \tabularnewline
            \texttt{rqt\_reconfigure} & Live configuration of nodes \tabularnewline
            \texttt{rqt\_diagnostics} & Standardized diagnostics \tabularnewline
            \texttt{rosgraph} & plots the node network \tabularnewline
            + tons of introspection tools & Print out/publish/call messages, services, nodes \tabularnewline
            \bottomrule
        \end{tabularx}
        \label{tab:options}
    \end{table}
\end{frame}

\section{ROS vs YARP}


\begin{frame}{Getting the terminology right}
    \begin{table}[]
        \begin{tabularx}{\linewidth}{l>{\raggedright}X}
            \toprule
            \textbf{YARP}			& \textbf{ROS} \tabularnewline
            \midrule
            \texttt{yarpserver}		& \texttt{roscore} \tabularnewline
            \midrule
            \texttt{bottle}		& \texttt{message} \tabularnewline
            \texttt{noserifmath}		& \texttt{topic} \tabularnewline
            \texttt{nosectionpages} & \texttt{action} (async) or \texttt{service} (sync) \tabularnewline
            \bottomrule
        \end{tabularx}
        \label{tab:options}
    \end{table}
\end{frame}


\end{document}






